\section{Related Work}

\textbf{Pub-Sub-Notify model in HPC application:}
The event based architecture has advantages in decouple for the aspects of time, space and synchronization\cite{eugster2003many}. Those advantages are reflected in cloud computing area and new service type such as FaaS (Function as a Service)\cite{fox2017status}. There are also all kinds of implementation in research and industry using event message broker based on those pub-sub-notify architecture\cite{hivemq,jin2012scalable}. However, these works did not fully use the properties of event driven architecture in terms of dynamical event matching and using event message to connect and trigger tasks in workflow dynamically. 


\textbf{Workflow Execution Framework for HPC application:}
There are two properties for the applications in dynamic workflow: the type of data intensive and the dependency between the multi-steps in workflow. There are all kinds of workflow manage tools and programming models aiming to simplify the abstraction of workflow during programming, such as Swift\cite{wilde2011swift} and Decaf \cite{dreher2017decaf} . For these types of systems and programming models, the developer should grasp the execution topology of the workflow in advance and excute the workflow by submitting the description file of task execution topology to the workflow engine. Even though those methods could cover lots of use contexts, but one drawback is the lack of data dynamics. One similar work Meteor\cite{jiang2008meteor} providing a content-based event driven decouple infrastructure based on p2p network which shows the flexibility for the dynamic programming event matching and client actions triggering. 

\textbf{Workflow Control Pattern:}
The Canonical workflow pattens are summarized in different contexts\cite{russell2006workflow} , for the HPC applications, the complex workflow could be constructed on several major patterns\cite{bharathi2008characterization}. The content-based event driven architecture could support those pattern and make some of them more efficient and dynamic in execution(supplement further)

\textbf{Runtime of Workflow Execution in HPC:}
There are all kinds of runtime frameworks including Charm++, Legion, OCR, Uintah can help researcher to express the application into AMT(asynchronised many task) model. The comparison including the prons and cons are also discussed in quantity.\cite{wilke2015asynchronousa,wilke2015asynchronousb},The applications and workflow can be programmed in sequential way and executed asynchronised by the scheduler provided by those runtime system, several parallel programming model even programming languages are further explored based on those runtime tools and AMT model.\cite{pebay2016towards,acun2014parallel,insituvisual}. Some works \cite{sun2016staging} are focusing on how to optimize the scheduler strategies to improve the running efficiency of in-situ workflow(supplement further), 
